% =============================================================================
%% Slide Templates, Tools and Modifiers
% This section is designed to provide simple commands which users can
% issue in order to create slides with the same visual appearance, and
% also to configure important working aspects:

%% This will create a cover slide:
\newcommand{\SlideCover}{{
            \usebackgroundtemplate{\includegraphics[width=\paperwidth]{./03_images/template/IPT_Cover.png}}
            \color{fgcolor}
            \setbeamertemplate{footline}{}
            \begin{frame}
                \vspace*{12\baselineskip}
                \huge{\textbf\thstitle}\par
                \smallskip
                \Large{\thsprof}\par
                \bigskip
                \large{\textbf\thstask}\par
                \normalsize{\thsstudent}\par
                \bigskip
                \bigskip
                \small{\thscity\space--\space\thsmonth, \thsyear}\par
                \bigskip
            \end{frame}}
}

%% This will create the default slide for general purpose content:

\newcommand{\SlideTextA}[2]{{
            \usebackgroundtemplate{\includegraphics[width=\paperwidth]{./03_images/template/IPT_Default_Slide.png}}
            \color{fgcolor}
            \begin{frame}
                \frametitle{#1}
                #2
            \end{frame}}
}

%% This slide will receive an image as a parameter and use it as background:
\newcommand{\SlideGBImage}[3]{{
            \usebackgroundtemplate{\includegraphics[width=\paperwidth]{#1}}
            \color{bgcolor}
            \begin{frame}
                \frametitle{#2}
                #3
            \end{frame}}
}

%% This will create a References section, splitting the content in multiple slides if necessary:
\newcommand{\SlideRefs}[2]{{
            \usebackgroundtemplate{\includegraphics[width=\paperwidth]{./03_images/template/IPT_Default_Slide.png}}
            \color{fgcolor}
            \begin{frame}[allowframebreaks]
                \frametitle{#1}
                \renewcommand*{\bibfont}{\footnotesize}
                #2
            \end{frame}}
}

%% This will create a closing slide, with the big IPT logo:
\newcommand{\SlideEnd}{{

            \setbeamercolor{background canvas}{bg=fgcolor}
            \color{bgcolor}
            \setbeamertemplate{footline}{}
            \begin{frame}
                \centering
                \includegraphics[width=100px]{./03_images/template/IPT_logo_blue_square.png}

                \large{OBRIGADO!}
            \end{frame}}
}

%% This will position the titles at the center of the slides:
\setbeamertemplate{frametitle}{
    \begin{center}
        \bigskip
        \textbf
        \insertframetitle
    \end{center}
}

%% This will proper format the figure captions:
% \ABNTfigure{caption}{reference}{path}{size}{label}
\newcommand{\ABNTfigure}[5]{
    \begin{figure}[ht]
        \centering
        \caption{#1}
        \includegraphics[#4]{#3}
        \caption*{\textbf{Fonte:} #2}
        \label{#5}
    \end{figure}
}

%% This will proper format the figure captions, and also
% expand the image to fill the maximum width and height available:
% \ABNTfigureAuto{caption}{reference}{path}{label}
\newcommand{\ABNTfigureAuto}[4]{

    \begin{figure}[!htb]
        \centering
        \vspace*{10pt}
        \caption{#1}
        \includegraphics[width=0.9\textwidth,height=0.57\textheight,keepaspectratio]{#3}
        \caption*{\textbf{Fonte:} #2}
        \label{#4}
    \end{figure}
}

%% This will implement boards, that are like tables, but
% used mainly for qualitative data:
\newcounter{boards}
\newcommand{\listboardname}{Lista de Quadros}
\newcommand{\board}[1] {
    \refstepcounter{boards}
    \par\noindent\textbf{Quadro \theboards\space--\space}#1
    \addcontentsline{lob}{boards}
    {\protect\numberline{\theboards}#1}\par
}

%% Now the boards environment is created, this shortcut will
% make it easier to use them:
% \ABNTboard{caption}{reference}{label}{tabular}
\newcommand{\ABNTboard}[4]{
    \begin{table}[ht]
        \centering
        \board{#1}
        \label{#3}
        #4
        \caption*{\textbf{Fonte:} #2}
    \end{table}
}

%% This will create a nice table of contents before each section:
\AtBeginSection[]
{
    \begingroup
    \SlideTextA{Conteúdo}{
        \tableofcontentscurrent
    }
    \endgroup
}
\setbeamertemplate{section in toc}{
    \color{fgcolor}
    $\blacktriangleright$ \inserttocsection
}
\setbeamertemplate{subsection in toc}{
    \color{fgcolor}
    \hspace*{10pt}$\circ$ \inserttocsubsection\par
}

% =============================================================================
