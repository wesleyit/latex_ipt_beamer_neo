% =============================================================================
%% Themes, Look and Feel
% This section is designed to define the presentation look and feel.
% Aspects like colors, font sizes and families are defined here:


%% Slide margins:
\setbeamersize{text margin left=25pt,text margin right=25pt}


%% Font Definitions:
\setmainfont{Bookerly Display}
\setsansfont{Segoe UI}
\setmonofont{FantasqueSansMono Nerd Font Mono}
\usefonttheme[onlymath]{serif}


%% This will define colors that can be used as a theme:
\definecolor{fgcolor}{HTML}{005B90}
\definecolor{bgcolor}{HTML}{FFFFFF}
\definecolor{c1color}{HTML}{00BAEA}
\definecolor{c2color}{HTML}{6E6E70}
\definecolor{c3color}{HTML}{F0F0F6}


%% Thee following sizes are used in the cover slide:
\setbeamerfont{title}{size=\Huge}  % Title slides
\setbeamerfont{subtitle}{size=\Large}
\setbeamerfont{author}{size=\normalsize}
\setbeamerfont{institute}{size=\large}
\setbeamerfont{date}{size=\tiny}
\setbeamerfont{footline}{size=\tiny}


%% For regular slides, the title is defined as follows:
\setbeamertemplate{frametitle}{
    \begin{minipage}[c][3.6em][c]{0.84\textwidth}
        \bigskip
        \textbf{\insertframetitle}\insertframesubtitle
    \end{minipage}
}
\setbeamerfont{frametitle}{size=\Large}
\setbeamerfont{framesubtitle}{size=\small}
\setbeamercolor{frametitle}{fg=bgcolor}
\setbeamercolor{frametitle-left}{fg=bgcolor}


%% Colors for slide components:
\setbeamercolor*{background canvas}{bg=bgcolor}  % Colors
\setbeamercolor*{title}{fg=fgcolor}
\setbeamercolor*{subtitle}{fg=fgcolor}
\setbeamercolor*{author}{fg=fgcolor}
\setbeamercolor*{footline}{fg=c2color}
\setbeamercolor*{itemize item}{fg=fgcolor}
\setbeamercolor*{itemize subitem}{fg=fgcolor}
\setbeamercolor*{normal text}{fg=fgcolor}
\setbeamercolor{bibliography item}{fg=c1color}
\setbeamercolor{bibliography entry author}{fg=fgcolor}
\setbeamercolor{bibliography entry title}{fg=fgcolor}
\setbeamercolor{bibliography entry location}{fg=fgcolor}
\setbeamercolor{bibliography entry note}{fg=fgcolor}


%% Style for other elements:
\setbeamertemplate{bibliography item}{}
\setbeamertemplate{navigation symbols}{}
\setbeamertemplate{footline}[frame number]


%% Include the number in captions:
\setbeamertemplate{caption}[numbered]
\setbeamertemplate{caption label separator}{\space--\space}
\captionsetup[figure]{labelfont=bf, name=Figura}


% =============================================================================
%% In Computer Science it is very common to present source code.
% This settings will format lstlistings properly:
\lstdefinestyle{ABNTCodeStyle}{
    backgroundcolor=\color{white},
    commentstyle=\itshape\color{darkgray},
    keywordstyle=\bfseries\color{magenta},
    numberstyle=\color{lightgray},
    stringstyle=\color{blue},
    basicstyle=\ttfamily\footnotesize\color{black},
    breakatwhitespace=false,
    breaklines=true,
    captionpos=t,
    frame=lines,
    identifierstyle=\bfseries\color{black},
    keepspaces=true,
    numbers=left,
    numbersep=20pt,
    showspaces=false,
    showstringspaces=false,
    showtabs=false,
    tabsize=2,
}
\lstset{style=ABNTCodeStyle}
\renewcommand{\lstlistingname}{Código-fonte}


% =============================================================================
%% Package hyperref setup
% This will set metadata in the resulting PDF file.
% It also will properly set colors in the hyperlinks:
\makeatletter
\pdfstringdefDisableCommands{
\def\\{}
\def\texttt#1{<#1>}
}
\makeatother
\hypersetup{
    colorlinks=true,
    linkcolor=fgcolor,
    filecolor=fgcolor,
    citecolor=fgcolor,
    urlcolor=c1color,
    pdfauthor={\thsstudent},
    pdftitle={\thstitle},
    pdfsubject={\thstitle},
    pdfkeywords={\thskeywords},
    pdfpagemode=FullScreen
}
%% This will create links to references and URLs
% and also display in which page a reference is cited:
\DefineBibliographyStrings{brazilian}{
    backrefpage={Citado na página\space},
    backrefpages={Citado nas páginas\space}
}

% =============================================================================
